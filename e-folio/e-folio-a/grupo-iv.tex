\grupo{}

\begin{align*}
	A =
	\begin{bmatrix*}[r]
		1  & i  & 0\\
		-i & -1 & 0\\
		0  & 0  & 3
	\end{bmatrix*}
	\in \mathcal{M}_{3 \times 3} (\mathbb{C})
\end{align*}

\begin{align*}
	\intertext{Pretende-se determinar todas as matrizes em
	$\mathcal{M}_{3 \times 3}(\mathbb{C})$ que comutam com $A$, ou seja:}
	X &\in \mathcal{M}_{3 \times 3}(\mathbb{C}): XA = AX
	\intertext{Pelo que vamos definir uma dada matriz $X$ e calcular $XA$ e $AX$:}
	X
	&=
	\begin{bmatrix*}[r]
		U\\
		V\\
		W
	\end{bmatrix*}
	=
	\begin{bmatrix*}[r]
		u_1 & u_2 & u_3 \\
		v_1 & v_2 & v_3 \\
		w_1 & w_2 & w_3
	\end{bmatrix*}\\
	XA
	&=
	\begin{bmatrix*}[r]
		u_1 & u_2 & u_3 \\
		v_1 & v_2 & v_3 \\
		w_1 & w_2 & w_3
	\end{bmatrix*}
	\begin{bmatrix*}[r]
		1  & i  & 0\\
		-i & -1 & 0\\
		0  & 0  & 3
	\end{bmatrix*}
	=
	\begin{bmatrix*}[r]
		u_1 - iu_2 & iu_1 - u_2 & 3u_3 \\
		v_1 - iv_2 & iv_1 - v_2 & 3v_3 \\
		w_1 - iw_2 & iw_1 - w_2 & 3w_3
	\end{bmatrix*}\\
	AX
	&=
	\begin{bmatrix*}[r]
		1  & i  & 0\\
		-i & -1 & 0\\
		0  & 0  & 3
	\end{bmatrix*}
	\begin{bmatrix*}[r]
		u_1 & u_2 & u_3 \\
		v_1 & v_2 & v_3 \\
		w_1 & w_2 & w_3
	\end{bmatrix*}
	=
	\begin{bmatrix*}[r]
		u_1 + iv_1   & u_2 + iv_2   & u_3 + iv_3 \\
		-iu_1 - v_1 & -iu_2 - v_2 & -iu_3 - v_3\\
		3w_1        & 3w_2        & 3w_3
	\end{bmatrix*}
	\intertext{
		Tem-se  que $XA - AX$:}
	XA - AX
	&= 
	\begin{bmatrix*}[r]
		u_1 - iu_2 & iu_1 - u_2 & 3u_3 \\
		v_1 - iv_2 & iv_1 - v_2 & 3v_3 \\
		w_1 - iw_2 & iw_1 - w_2 & 3w_3
	\end{bmatrix*}
	-
	\begin{bmatrix*}[r]
		u_1 + iv_1   & u_2 + iv_2   & u_3 + iv_3 \\
		-iu_1 - v_1 & -iu_2 - v_2 & -iu_3 - v_3\\
		3w_1        & 3w_2        & 3w_3
	\end{bmatrix*}\\
	&=
	\begin{bmatrix*}[r]
	- i u_{2} - i v_{1}         & i u_{1} - 2 u_{2} - i v_{2} & 2 u_{3} - i v_{3}\\
	i u_{1} + 2 v_{1} - i v_{2} & i u_{2} + i v_{1}           & i u_{3} + 4 v_{3}\\
	- 2 w_{1} - i w_{2}         & i w_{1} - 4 w_{2}           & 0
	\end{bmatrix*}
\end{align*}

\paragraph{}Dado que $XA = AX \iff XA - AX = 0_{3 \times 3}$, podemos agora
determinar os vetores $U, V, W$, com base no resultado anterior.
Para $W$, isolamos o sistema da última linha, dado que só nessa é que tem
dependência, e pelo qual já concluímos que $w_3$ não tem restrições.

\begin{align*}
	&
	\begin{bmatrix*}
		-2 & -i\\
		i & -4
	\end{bmatrix*}
	\begin{bmatrix}
		w_1\\
		w_2
	\end{bmatrix}
	=
	0_{2 \times 1}\\
	\det &
	\begin{bmatrix*}
		-2 & -i\\
		i & -4
	\end{bmatrix*}
	= -2 \cdot (-4) - (-i) \cdot i = 7 \neq 0
\end{align*}

\paragraph{} Dado que o determinante da matriz simples não é nulo, podemos
concluir que o sistema é determinado. Dado que o sistema é homogéneo, tem-se
forçosamente de concluir que, dado que pela regra de Cramer teríamos colunas
nulas, $(w_1, w_2) = (0, 0)$,
pelo que se tem $W = (0, 0, w_3)$

\paragraph{}Agora determinamos $u_3$ e $v_3$ pelas equações da última coluna.
Tem-se que
\begin{align*}
	&
	\begin{bmatrix*}
		2 & -i\\
		i & 4
	\end{bmatrix*}
	\begin{bmatrix}
		u_3\\
		v_3
	\end{bmatrix}
	=
	0_{2 \times 1}\\
	\det &
	\begin{bmatrix*}
		2 & -i\\
		i & 4
	\end{bmatrix*}
	= 2 \cdot 4 - (-i) \cdot i = 7 \neq 0
\end{align*}

\paragraph{} De forma análoga a $(w_1, w_2)$, tem-se que $(u_3, v_3) = (0, 0)$.

\paragraph{} Determinemos agora $(u_1, u_2, v_1, v_2)$.

\begin{align*}
	\begin{cases}
		- i u_2 - i v_1 = 0\\
		i u_1 - 2 u_2 - i v_2 = 0\\
		i u_1 + 2 v_1 - i v_2 = 0\\
		i u_2 + i v_1 = 0
	\end{cases}
	\iff
	\begin{cases}
		u_2 = - v_1\\
		i u_1 - 2 u_2 - i v_2 = 0\\
		i u_1 + 2 v_1 - i v_2 = 0\\
	\end{cases}
	\iff
	\begin{cases}
		u_2 = - v_1\\
		i u_1 + 2 v1 - i v_2 = 0\\
	\end{cases}
	\iff
	\begin{cases}
		u_2 = - v_1\\
		u_1 = -2v_1 + iv_2\\
	\end{cases}
\end{align*}

\paragraph{Por completar}







