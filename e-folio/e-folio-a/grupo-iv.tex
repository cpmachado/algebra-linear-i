\grupo{}

\begin{align*}
	A =
	\begin{bmatrix*}[r]
		1  & i  & 0\\
		-i & -1 & 0\\
		0  & 0  & 3
	\end{bmatrix*}
	\in \mathcal{M}_{3 \times 3} (\mathbb{C})
\end{align*}

\begin{align*}
	\intertext{Pretende-se determinar todas as matrizes em
	$\mathcal{M}_{3 \times 3}(\mathbb{C})$ que comutam com $A$, ou seja:}
	X &\in \mathcal{M}_{3 \times 3}(\mathbb{C}): XA = AX
	\intertext{Pelo que vamos definir uma dada matriz $X$ e calcular $XA$ e $AX$:}
	X
	&=
	\begin{bmatrix*}[r]
		U\\
		V\\
		W
	\end{bmatrix*}
	=
	\begin{bmatrix*}[r]
		u_1 & u_2 & u_3 \\
		v_1 & v_2 & v_3 \\
		w_1 & w_2 & w_3
	\end{bmatrix*}\\
	XA
	&=
	\begin{bmatrix*}[r]
		u_1 & u_2 & u_3 \\
		v_1 & v_2 & v_3 \\
		w_1 & w_2 & w_3
	\end{bmatrix*}
	\begin{bmatrix*}[r]
		1  & i  & 0\\
		-i & -1 & 0\\
		0  & 0  & 3
	\end{bmatrix*}
	=
	\begin{bmatrix*}[r]
		u_1 + iu_2 & iu_1 - u_2 & 3u_3 \\
		v_1 + iv_2 & iv_1 - v_2 & 3v_3 \\
		w_1 + iw_2 & iw_1 - w_2 & 3w_3
	\end{bmatrix*}\\
	AX
	&=
	\begin{bmatrix*}[r]
		1  & i  & 0\\
		-i & -1 & 0\\
		0  & 0  & 3
	\end{bmatrix*}
	\begin{bmatrix*}[r]
		u_1 & u_2 & u_3 \\
		v_1 & v_2 & v_3 \\
		w_1 & w_2 & w_3
	\end{bmatrix*}
	=
	\begin{bmatrix*}[r]
		u_1 + iv1   & u_2 + iv2   & u_3 + iv3 \\
		-iu_1 - v_1 & -iu_2 - v_2 & -iu_3 - v_3\\
		3w_1        & 3w_2        & 3w_3
	\end{bmatrix*}
	\intertext{Dado que $XA = AX \iff XA - AX = 0_{3 \times 3}$,
		vamos determinar $XA - AX$:}
	XA - AX
	&= 
	\begin{bmatrix*}[r]
		u_1 + iu_2 & iu_1 - u_2 & 3u_3 \\
		v_1 + iv_2 & iv_1 - v_2 & 3v_3 \\
		w_1 + iw_2 & iw_1 - w_2 & 3w_3
	\end{bmatrix*}
	-
	\begin{bmatrix*}[r]
		u_1 + iv1   & u_2 + iv2   & u_3 + iv3 \\
		-iu_1 - v_1 & -iu_2 - v_2 & -iu_3 - v_3\\
		3w_1        & 3w_2        & 3w_3
	\end{bmatrix*}\\
	&=
	\begin{bmatrix*}[r]
		iu_2 - iv_1        & iu_1 - 2u_2 - iv_2 & 2u_3 -iv_3 \\
		2v_1 + iv_2 + iu_1 & iv_1 + iu_2        & 4v_3 + iu_3 \\
		-2w_1 + iw_2       & iw_1 - 4w_2        & 0
	\end{bmatrix*}
\end{align*}

\begin{align*}
	\intertext{Pelo que todo o elemento de $XA - AX$ tem de igualar a 0, e as
		suas dependências, podemos observar que a última linha apenas depende
		do vector $W$ e as duas primeiras linhas de $U$ e $V$, pelo que
	podemos resolver os problemas separadamente, como se segue:}
	\begin{bmatrix*}
		0 & i  & 0 & -i & 0  & 0\\
		i & -2 & 0 & 0  & -i & 0\\
		0 & 0  & 2 & 0  & 0  & -i\\
		i & 0  & 0 & 2  & i  & 0\\
		0 & i  & 0 & i  & 0  & 0\\
		0 & 0  & i & 0  & 0  & 4
	\end{bmatrix*}
	&\overset{
		\begin{matrix}
			l_1 \leftrightarrow l_4\\
			l_3 \leftrightarrow l_5\\
		\end{matrix}
	}{\longrightarrow}
	\begin{bmatrix*}
		i & 0  & 0 & 2  & i  & 0\\
		i & -2 & 0 & 0  & -i & 0\\
		0 & i  & 0 & i  & 0  & 0\\
		0 & i  & 0 & -i & 0  & 0\\
		0 & 0  & 2 & 0  & 0  & -i\\
		0 & 0  & i & 0  & 0  & 4
	\end{bmatrix*}
	\overset{
		\begin{matrix}
			l_2 - l_1\\
			l_4 - l_3\\
			l_6 - \frac{1}{2}il_5
		\end{matrix}
	}{\longrightarrow}
	\begin{bmatrix*}
		i & 0  & 0 & 2  & i  & 0\\
		0 & -2 & 0 & -2  & -2i & 0\\
		0 & i  & 0 & i  & 0  & 0\\
		0 & 0  & 0 & -2i & 0  & 0\\
		0 & 0  & 2 & 0  & 0  & -i\\
		0 & 0  & 0 & 0  & 0  & \frac{9}{2}
	\end{bmatrix*}\\
	\overset{
		\begin{matrix}
			-il_1\\
			-2^{-1}l_2\\
			-il_3
		\end{matrix}
	}{\longrightarrow}
	\begin{bmatrix*}
		1 & 0  & 0 & -2i  & 1  & 0\\
		0 & 1 & 0 & 1  & i & 0\\
		0 & 1  & 0 & 1  & 0  & 0\\
		0 & 0  & 0 & -2i & 0  & 0\\
		0 & 0  & 2 & 0  & 0  & -i\\
		0 & 0  & 0 & 0  & 0  & \frac{9}{2}
	\end{bmatrix*}&
	\overset{
		\begin{matrix}
			2^{-1}il_4\\
			2^{-1}l_5\\
			2 \cdot 9^{-1}l_6
		\end{matrix}
	}{\longrightarrow}
	\begin{bmatrix*}
		1 & 0 & 0 & -2i & 1 & 0\\
		0 & 1 & 0 & 1   & i & 0\\
		0 & 1 & 0 & 1   & 0 & 0\\
		0 & 0 & 0 & 1   & 0 & 0\\
		0 & 0 & 1 & 0   & 0 & -\frac{1}{2}i\\
		0 & 0 & 0 & 0   & 0 & 1
	\end{bmatrix*}
	\overset{
		\begin{matrix}
			l_2 - l_3\\
			l_3 - l_4\\
			l_5 + 2^{-1}il_6
		\end{matrix}
	}{\longrightarrow}
	\begin{bmatrix*}
		1 & 0 & 0 & -2i & 1 & 0\\
		0 & 0 & 0 & 0   & i & 0\\
		0 & 1 & 0 & 0   & 0 & 0\\
		0 & 0 & 0 & 1   & 0 & 0\\
		0 & 0 & 1 & 0   & 0 & 0\\
		0 & 0 & 0 & 0   & 0 & 1
	\end{bmatrix*}\\
	\overset{
		\begin{matrix}
			l_1 + il_2 + 2il_4\\
			-il_2
		\end{matrix}
	}{\longrightarrow}
	\begin{bmatrix*}
		1 & 0 & 0 & 0 & 0 & 0\\
		0 & 0 & 0 & 0   & 1 & 0\\
		0 & 1 & 0 & 0   & 0 & 0\\
		0 & 0 & 0 & 1   & 0 & 0\\
		0 & 0 & 1 & 0   & 0 & 0\\
		0 & 0 & 0 & 0   & 0 & 1
	\end{bmatrix*}&
\end{align*}
\begin{align*}
	\intertext{Que nos permite concluir que}
	&U = V = 
	\begin{bmatrix*}
		0 & 0 & 0
	\end{bmatrix*}
	\intertext{Vamos definir $W$}
	\begin{bmatrix*}
		-2  & i & 0\\
		i & -4 & 0
	\end{bmatrix*}
	&\overset{l_2 + \frac{i}{2}l_1}{\longrightarrow}
	\begin{bmatrix*}
		-2  & i & 0\\
		0 & -\frac{9}{2} & 0
	\end{bmatrix*}
	\overset{-\frac{2}{9}l_2}{\longrightarrow}
	\begin{bmatrix*}
		-2  & i & 0\\
		0 & 1 & 0
	\end{bmatrix*}
	\overset{-\frac{1}{2} l_1 + \frac{1}{2}il_2}{\longrightarrow}
	\begin{bmatrix*}
		1  & 0 & 0\\
		0 & 1 & 0
	\end{bmatrix*}
	\intertext{Que nos permite concluir que}
	&W =
	\begin{bmatrix}
		0 & 0 & w_3
	\end{bmatrix}
	\intertext{E que a matriz $X$ é definida por}
	X &= 
	\begin{bmatrix}
		0 & 0 & 0\\
		0 & 0 & 0\\
		0 & 0 & w_3
	\end{bmatrix}
	\intertext{com $w_3 \in \mathbb{C}$}
\end{align*}

