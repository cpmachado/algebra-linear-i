\grupo{}

\paragraph{} 


\begin{align*}
	\intertext{Sejam $n\geq 1$, $A \in \mathcal{M}_{n \times n}(\mathbb{R})$,
		tal que $A$ tem a forma}
	A =
	\begin{bmatrix*}[r]
		0 & 0 & \ldots & 0 & 0 & a_1\\
		0 & 0 & \ldots & 0 & a_2 & *\\
		0 & 0 & \ldots & a_3 & * & *\\
		\vdots & \vdots & \iddots & \vdots & \vdots & \vdots\\
		0 & a_{n -1} & \ldots & * & * & *\\
		a_n & * & \ldots & * & * & *
	\end{bmatrix*}
	\intertext{onde $a_1, a_2, \ldots, a_n \in \mathbb{R}$ e os * representam
		também valores arbitrários de $\mathbb{R}$, que podem ou não ser
	distintos.}
\end{align*}

\begin{proposition}
	$\det A = (-1)^p \prod_{k=1}^{n} a_k$, com $p = \lfloor
	\frac{n}{2}\rfloor$.
\end{proposition}

\begin{proof}\; \\
	Pode-se observar que a matriz $A$ é equivalente por linhas com uma
	matriz triangular superior $B$, por sucessivas transformações lineares de
	$l_i \leftrightarrow l_{n - i}$, com $i = 1, \ldots, \lfloor \frac{n}{2}\rfloor$. Pela Proposição 3.15 da
	bibliografia\footcite[pág. 135, Proposição 3.15]{Cabral2012}, tem-se que
	$\det B = \prod_{k=1}^{n} b_k$, o produto dos elementos da diagonal principal de $B$. A
	partir da Proposição 3.19, alínea 1., da bibliografia
	\footcite[pág. 138, Proposição 3.19, alínea 1.]{Cabral2012}, tem-se que
	cada mudança de linhas o determinante é igual ao seu simétrico.
	Deste modo, com $p = \lfloor\frac{n}{2}\rfloor$ transformações lineares
	desse tipo, podemos concluir que $\det A = (-1)^p \prod_{k=1}^{n}
	a_k$.\\
\end{proof}
