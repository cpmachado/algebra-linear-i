\grupo{}

\paragraph{} 


\begin{align*}
	\intertext{Sejam $n\geq 1$, $A \in \mathcal{M}_{n \times n}(\mathbb{R})$,
		tal que $A$ tem a forma}
	A =
	\begin{bmatrix*}[r]
		0 & 0 & \ldots & 0 & 0 & a_1\\
		0 & 0 & \ldots & 0 & a_2 & *\\
		0 & 0 & \ldots & a_3 & * & *\\
		\vdots & \vdots & \iddots & \vdots & \vdots & \vdots\\
		0 & a_{n -1} & \ldots & * & * & *\\
		a_n & * & \ldots & * & * & *
	\end{bmatrix*}
	\intertext{onde $a_1, a_2, \ldots, a_n \in \mathbb{R}$ e os * representam
		também valores arbitrários de $\mathbb{R}$, que podem ou não ser
	distintos.}
\end{align*}

\begin{proposition}
	$|\det A| = \prod_{k=1}^{n} |a_k|$
\end{proposition}

\begin{proposition}
	Seja $A \in \mathcal{M}$ $\det A = \prod_{k=1}^{n} a_k$
\end{proposition}

\begin{proposition}
	$\det A = \pm \prod_{k=1}^{n} a_k$
\end{proposition}

\begin{proof}
	\; \\
	\begin{align*}
		a_{ij} =
		\begin{cases}
			0, & j < n - i\\
			a_i, & j = n - i \\
			*, & j > n - i
		\end{cases}
	\end{align*}
	Considerar uma linha, e fazer a definição recursiva até 1.
\end{proof}
