\grupo{}

\exercicio{}

\paragraph{\underline{Resposta}}

\paragraph{}É a opção d). $CB$ é invertível.

\paragraph{\underline{Justificação}}

\begin{align*}
	\intertext{Tem-se que as matrizes $A$, $B$ e $C$ são definidas por}
	A =
	\begin{bmatrix*}[r]
		2  & 3\\
		2  & 0\\
		-3 & 1
	\end{bmatrix*}
	\in
	\mathcal{M}_{3 \times 2}(\mathbb{K})
	,\;
	B =
	\begin{bmatrix*}[r]
		-3\\
		1
	\end{bmatrix*}
	\in
	\mathcal{M}_{2 \times 1}(\mathbb{K})
	 \text{ e }
	C =
	\begin{bmatrix*}[r]
		1  & 2
	\end{bmatrix*}
	\in
	\mathcal{M}_{1 \times 2}(\mathbb{K})
	.
\end{align*}

\begin{proposition}[negação da opção 1.a]\label{prop:i-1-a}
	$AB$ é definido.
\end{proposition}

\begin{proof}
	\; \\
	Pela definição da matriz produto
	\footcite[pág. 12, Definição 1.18: matriz produto]{Cabral2012},
	esta só é definida se o número de colunas do
	operando à esquerda for igual ao número de linhas do operando à
	direita. Como
	$A$ tem o mesmo número de colunas que o número de linhas de $B$,
	então $AB$ é definido.\\
\end{proof}

\begin{proposition}[negação da opção 1.b]\label{prop:i-1-b}
	$AC$ não é definido.
\end{proposition}

\begin{proof}
	\; \\
	Pela definição da matriz produto
	\footcite[pág. 12, Definição 1.18: matriz produto]{Cabral2012},
	esta só é definida se o número de colunas do
	operando à esquerda for igual ao número de linhas do operando à
	direita. Como
	$A$ tem um número de colunas diferente do número de linhas de $C$,
	então $AC$ não é definido.\\
\end{proof}

\begin{proposition}[negação da opção 1.c]\label{prop:i-1-c}
	$(BC)^2$ não é invertível.
\end{proposition}

\begin{proof}
	\; \\
	Pela definição da matriz produto
	\footcite[pág. 12, Definição 1.18: matriz produto]{Cabral2012},
	tem-se que:
	\begin{align*}
		BC &=
		\begin{bmatrix*}[r]
			-3\\
			1
		\end{bmatrix*}
		\begin{bmatrix*}[r]
			1  & 2
		\end{bmatrix*}
		=
		\begin{bmatrix*}[r]
			-3  & -6\\
			1  & 2\\
		\end{bmatrix*}
		\intertext{
			Dado que as linhas de $BC$ não são linearmente
			independentes, e que a operação $l_i + \beta l_j$
			não tem impacto no determinante
			\footcite[pág. 138, Proposição 3.19, alínea 3.]{Cabral2012},
			tem-se que:}
			BC&
		\overset{l_1 + 3 l_2}{\longrightarrow}
		\begin{bmatrix*}[r]
			0  & 0\\
			1  & 2\\
		\end{bmatrix*}
		\implies \det(BC) = 0
		\intertext{
			Dado que o determinante da matriz produto é igual ao produto dos
			determinantes do operandos
			\footcite[pág. 145, Proposição 3.24]{Cabral2012}, tem-se que:
		}
		\det((BC)^2) = (\det(BC))^2 = 0
	\end{align*}
	Dado que a matriz só é invertível, se o determinante não for nulo
	\footcite[pág. 144, Proposição 3.23]{Cabral2012}, concluímos que
	$(BC)^2$ não é invertível.
\end{proof}

\clearpage
\begin{proposition}[opção 1.d]\label{prop:i-1-d}
	$CB$ é invertível.
\end{proposition}

\begin{proof}
	Pela definição da matriz produto
	\footcite[pág. 12, Definição 1.18: matriz produto]{Cabral2012},
	tem-se que:
	\begin{align*}
		CB =
		\begin{bmatrix*}
			1 & 2
		\end{bmatrix*}
		\begin{bmatrix*}[r]
			-3\\
			1
		\end{bmatrix*}
		=
		\begin{bmatrix}
			-1
		\end{bmatrix} \implies \det (CB) = -1 \neq 0
	\end{align*}
	Dado que a matriz só é invertível, se o determinante não for nulo
	\footcite[pág. 144, Proposição 3.23]{Cabral2012}, concluímos que
	$CB$ é invertível.
\end{proof}

\vspace{0.5cm}
\hrule
\vspace{0.5cm}


\exercicio{}

\paragraph{\underline{Resposta}}

\paragraph{} É a opção b). $\det(-A(B + C)^{-1}) = 3i$.


\paragraph{\underline{Justificação}}


\paragraph{}Sejam $A, B, C \in \mathcal{M}_{4 \times 4}(\mathbb{C})$ tais que
$\det A = 3$, e $\det(B + C) = i$.

\begin{proposition}[negação da opção a]\label{prop:i-2-a}
	Não existe elementos para calcular $\det(A + B + C)$, pelo que não é
	possível afirmar que $\det(A + B + C) = 3 + i$.
\end{proposition}

\begin{proof}\;\\
	Tem-se definidos $\det A$ e $\det(B + C)$, no entanto pelo teorema etc,
	pelo que não implica que $\det A + \det(B + C) = \det(A + B + C)$.
	Pelo que a proposição \ref{prop:i-2-a} está demonstrada e a opção a) não
	está correcta, ou antes não pode ser deduzida dos
	factos conhecidos.\\
\end{proof}


\begin{proposition}[opção b]\label{prop:i-2-b}
	$\det(-A(B + C)^{-1}) = 3i$
\end{proposition}

\begin{proof}[Cálculo do determinante da Proposição \ref{prop:i-2-b}]
	\begin{align*}
		\det(-A(B + C)^{-1})
			   = (- \det A)  \cdot (\det(B + C))^{-1}
			   = -\frac{3}{i}
			   = -\frac{3}{i} \cdot \frac{i}{i}
			   = 3i
	\end{align*}
	Pelo que a proposicão \ref{prop:i-2-b} é verdadeira e a opção b) está correcta.\\
\end{proof}


\begin{proposition}[negação da opção c]
	Não existe elementos para calcular $\det(A^2 BC)$, pelo que não é
	possível afirmar que $\det(A^2 BC) = 9i$.
\end{proposition}

\begin{proposition}[negação da opção d]\label{prop:i-2-d}
	$\det(-(B + C)A^{-1}) \neq \frac{i}{3}$
\end{proposition}

\begin{proof}[Cálculo do determinante da Proposição \ref{prop:i-2-d}]
	\begin{align*}
		\det(-(B + C)A^{-1})
			   = (-\det(B + C))  \cdot (\det A)^{-1}
			   = -\frac{i}{3} \neq \frac{i}{3}
	\end{align*}
	Pelo que a opção d) não está correcta.\\
\end{proof}



