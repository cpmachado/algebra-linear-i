\grupo{}

\exercicio{}

\paragraph{Resposta:} É a opção d). $CB$ é invertível.



\begin{align*}
	CB =
	\begin{bmatrix*}
		1 & 2
	\end{bmatrix*}
	\begin{bmatrix*}[r]
		-3\\
		1
	\end{bmatrix*}
	=
	\begin{bmatrix}
		-1
	\end{bmatrix} \implies \det (CB) = -1 \neq 0
\end{align*}


\exercicio{}

\paragraph{Resposta:} É a opção b). $\det(-A(B + C)^{-1}) = 3i$.

\paragraph{}Sejam $A, B, C \in \mathcal{M}_{4 \times 4}(\mathbb{C})$ tais que
$\det A = 3$, e $\det(B + C) = i$.


\begin{align*}
	\det(-A(B + C)^{-1})
		   &= (- \det A)  \cdot (\det(B + C))^{-1}\\
		   &= -\frac{3}{i}
		   = -\frac{3}{i} \cdot \frac{i}{i}
		   = 3i
\end{align*}


