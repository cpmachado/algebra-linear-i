\exercicio{}

\paragraph{\underline{Resposta}}

\paragraph{} É a opção b) $\det(-A(B + C)^{-1}) = 3i$.

\paragraph{\underline{Dados do enunciado}}

\paragraph{}Sejam $A, B, C \in \mathcal{M}_{4 \times 4}(\mathbb{C})$ tais que
$\det A = 3$, e $\det(B + C) = i$.

\paragraph{\underline{Justificação}}


\paragraph{}Vamos analisar cada uma das opções sequencialmente.

\begin{enumerate}[label=\bfseries\alph*)]
	\item $\det(A + B + C) = 3 + i$\\
		Não é possível verificar a afirmação com base nos dados
		disponibilizados.
	\item $\det(-A(B+C)^{-1}) = 3i$\\
		É uma afirmação verdadeira, comprovada pelo cálculo do determinante,
		como segue:
		\begin{align*}
			\det(-A(B+C)^{-1})
			&= (-\det A)(\det(B+C))^{-1}\\
			&= (-3)\left(\frac{1}{i}\right) \frac{i}{i}
			= (-3)(-i)
			= 3i
		\end{align*}
	\item $\det(A^2 B C) = 9i$\\
		Não é possível verificar a afirmação com base nos dados
		disponibilizados. Pela propriedade dos determinantes, não é possível
		determinar 
	\item $\det(-(B+C)A^{-1}) = \frac{i}{3}$\\
		É uma afirmação falsa, comprovado pelo cálculo do determinante,
		como segue:
		\begin{align*}
			\det(-(B+C)A^{-1})
			&= (-\det(B+C))(\det A)^{-1}\\
			&= (-i)\left(\frac{1}{3}\right) = - \frac{i}{3}
			\neq \frac{i}{3}
		\end{align*}
\end{enumerate}
