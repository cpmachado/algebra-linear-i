\exercicio{}

\paragraph{\underline{Resposta}}

\paragraph{}É a opção d) $CB$ é invertível.

\paragraph{\underline{Dados do enunciado}}

\begin{align*}
	\intertext{Tem-se que as matrizes $A$, $B$ e $C$ são definidas por}
	A =
	\begin{bmatrix*}[r]
		2  & 3\\
		2  & 0\\
		-3 & 1
	\end{bmatrix*}
	\in
	\mathcal{M}_{3 \times 2}(\mathbb{K})
	,\;
	B =
	\begin{bmatrix*}[r]
		-3\\
		1
	\end{bmatrix*}
	\in
	\mathcal{M}_{2 \times 1}(\mathbb{K})
	 \text{ e }
	C =
	\begin{bmatrix*}[r]
		1  & 2
	\end{bmatrix*}
	\in
	\mathcal{M}_{1 \times 2}(\mathbb{K})
\end{align*}

\paragraph{\underline{Justificação}}

\paragraph{}Vamos analisar cada uma das opções sequencialmente.

\begin{enumerate}[label=\bfseries\alph*)]
	\item $AB$ não é definido.\\
		É uma afirmação falsa.
			A matriz $A$ tem o mesmo número de colunas que o número de
			linhas de $B$,
			pela definição de matriz produto
			\footcite[pág. 12, Definição 1.18]{Cabral2012}
			conclui-se que $AB$ é definido.
	\item $AC$ é definido.\\
		É uma afirmação falsa.
			A matriz $A$ não tem o mesmo número de colunas que o número de
			linhas de $C$,
			pela definição de matriz produto
			\footcite[pág. 12, Definição 1.18]{Cabral2012}
			conclui-se que $AC$ não é definido.
	\item $(BC)^2$ é invertível.\\
		É uma afirmação falsa. Calculando o determinante, verifica-se que é
		nulo, consequentemente, pela Proposição 3.25
			\footcite[pág. 145, Proposição 3.25]{Cabral2012},
		$(BC)^2$ não é invertível.
		\begin{align*}
			BC &=
			\begin{bmatrix*}[r]
				-3\\
				1
			\end{bmatrix*}
			\begin{bmatrix*}[r]
				1 & 2
			\end{bmatrix*}
			=
			\begin{bmatrix*}[r]
				-3 & -6\\
				1  & 2
			\end{bmatrix*}\\
			\implies
			\det(BC) &= -3 \cdot 2 - (-6) \cdot 1 = 0
			\intertext{Dado que produto do determinante de matrizes é
			equivalente ao determinante do produto das matrizes
			\footcite[pág. 145, Proposição 3.24]{Cabral2012}
		, tem-se}
			\iff \det((BC)^2) &= \left(\det(BC)\right)^2 = 0
		\end{align*}
	\item $CB$ é invertível.\\
		É uma afirmação verdadeira. Calculando o determinante, verifica-se que
		não é nulo, consequentemente, pela Proposição 3.25
			\footcite[pág. 145, Proposição 3.25]{Cabral2012},
		$CB$ é invertível.
		\begin{align*}
			CB =
			\begin{bmatrix*}[r]
				1 & 2
			\end{bmatrix*}
			\begin{bmatrix*}[r]
				-3\\
				1
			\end{bmatrix*}
			=
			\begin{bmatrix*}[r]
				-1
			\end{bmatrix*}
			\implies
			\det(CB) = -1 \neq 0
		\end{align*}
\end{enumerate}
