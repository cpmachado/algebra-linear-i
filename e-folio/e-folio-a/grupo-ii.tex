\grupo{}

\paragraph{\underline{Dados do enunciado}}
\paragraph{} Tem-se $A \in \mathcal{M}_{5 \times 5}(\mathbb{K})$ e
$b \in \mathcal{M}_{5 \times 1}(\mathbb{K})$, definidas por:

\begin{align*}
	A &=
	\begin{bmatrix*}[r]
		2 & 1 & 0 & 3 & 0\\
		3 & -1 & 1 & 4 & 0\\
		0 & 0 & 2 & 1 & 1\\
		0 & 0 & 3 & 0 & 1\\
		0 & 0 & 1 & 0 & -2
	\end{bmatrix*}
	\quad
	\text{ e }
	\quad
	b =
	\begin{bmatrix*}[r]
		0\\
		1\\
		2\\
		r_3\\
		0
	\end{bmatrix*},\\
	\text{com }r_3 &= \text{(número de aluno)} \mod 1000.
\end{align*}

\paragraph{\underline{Resolução}}

\begin{enumerate}[label=\alph*.]
	\item \;\\
		\textbf{\underline{Resposta}}: $\det A = -35$

		\textbf{\underline{Cálculo}}:

		Pretende-se calcular o determinante de $A$, pelo que se vai recorrer
		múltiplas vezes ao Teorema de Laplace
		\footcite[pág. 129, Proposiçãp 3.9, Teorema de Laplace]{Cabral2012},
		no qual se utiliza a notação mencionada na bibliografia da cadeira.
		\begin{align*}
			\det A
			&=
			\det
			\begin{bmatrix*}[r]
				2 & 1 & 0 & 3 & 0\\
				3 & -1 & 1 & 4 & 0\\
				0 & 0 & 2 & 1 & 1\\
				0 & 0 & 3 & 0 & 1\\
				0 & 0 & 1 & 0 & -2
			\end{bmatrix*}\\
			&
			\begin{matrix}
				\text{Lapl.}\\
				=\\
				c_1
			\end{matrix}
			2
			\det
			\begin{bmatrix*}[r]
				-1 & 1 & 4 & 0\\
				0 & 2 & 1 & 1\\
				0 & 3 & 0 & 1\\
				0 & 1 & 0 & -2
			\end{bmatrix*}
			-
			3
			\det
			\begin{bmatrix*}[r]
				1 & 0 & 3 & 0\\
				0 & 2 & 1 & 1\\
				0 & 3 & 0 & 1\\
				0 & 1 & 0 & -2
			\end{bmatrix*}\\
			&
			\begin{matrix}
				\text{Lapl.}\\
				=\\
				c_1
			\end{matrix}
			-2
			\det
			\begin{bmatrix*}[r]
				2 & 1 & 1\\
				3 & 0 & 1\\
				1 & 0 & -2
			\end{bmatrix*}
			-
			3
			\det
			\begin{bmatrix*}[r]
				2 & 1 & 1\\
				3 & 0 & 1\\
				1 & 0 & -2
			\end{bmatrix*}\\
			&=
			- 5
			\det
			\begin{bmatrix*}[r]
				2 & 1 & 1\\
				3 & 0 & 1\\
				1 & 0 & -2
			\end{bmatrix*}\\
			&
			\begin{matrix}
				\text{Lapl.}\\
				=\\
				l_1
			\end{matrix}
			- 5\left(
			2
			\det
			\begin{bmatrix*}[r]
				0 & 1\\
				0 & -2
			\end{bmatrix*}
			-
			\det
			\begin{bmatrix*}[r]
				3 & 1\\
				1 & -2
			\end{bmatrix*}
			+
			\det
			\begin{bmatrix*}[r]
				3 & 0\\
				1 & 0
			\end{bmatrix*}\right)\\
			&=
			-5[0 -[3 \times (-2) - 1 \times 1] + 0]\\
			&= -35
		\end{align*}
	\clearpage
	\item \;\\
		\textbf{\underline{Resposta}}: $x_4 = -\frac{4531}{7}\approx -647.29$

		\textbf{\underline{Dados do enunciado}}: A equação $AX=b$ tem uma
		solução única $X = (x_1, x_2, x_3, x_4, x_5)^T$.

		\textbf{\underline{Cálculo}}:

		Como $\det A = -35 \neq 0$, pode-se recorrer à
		regra de Cramer
		\footcite[pág. 152, Proposição 3.31, Regra de Cramer]{Cabral2012}
		para determinar $x_4$. Considerando que é a quarta varíavel da matriz
		de incógnitas, tem-se:
		\begin{align*}
			x_4 = \frac{\det A(4)}{\det A}
			&
			=
			\frac{
				\det
				\begin{bmatrix*}[r]
					2 & 1  & 0 & 0   & 0\\
					3 & -1 & 1 & 1   & 0\\
					0 & 0  & 2 & 2   & 1\\
					0 & 0  & 3 & r_3 & 1\\
					0 & 0  & 1 & 0   & -2
				\end{bmatrix*}
			}{
				(-35)
			}\\
			&
			\begin{matrix}
				\text{Lapl.}\\
				=\\
				c_1
			\end{matrix}
			-
			\frac{
				2
				\det
				\begin{bmatrix*}[r]
					-1 & 1 & 1   & 0\\
					0  & 2 & 2   & 1\\
					0  & 3 & r_3 & 1\\
					0  & 1 & 0   & -2
				\end{bmatrix*}
				-3
				\det
				\begin{bmatrix*}[r]
					1  & 0 & 0   & 0\\
					0  & 2 & 2   & 1\\
					0  & 3 & r_3 & 1\\
					0  & 1 & 0   & -2
				\end{bmatrix*}
			}{
				35
			}\\
			&
			\begin{matrix}
				\text{Lapl.}\\
				=\\
				c_1
			\end{matrix}
			-
			\frac{
				-2
				\det
				\begin{bmatrix*}[r]
					2 & 2   & 1\\
					3 & r_3 & 1\\
					1 & 0   & -2
				\end{bmatrix*}
				-3
				\det
				\begin{bmatrix*}[r]
					2 & 2   & 1\\
					3 & r_3 & 1\\
					1 & 0   & -2
				\end{bmatrix*}
			}{
				5 \cdot 7
			}\\
			&
			=
			\quad
			\frac{
				\det
				\begin{bmatrix*}[r]
					2 & 2   & 1\\
					3 & r_3 & 1\\
					1 & 0   & -2
				\end{bmatrix*}
			}{
				7
			}\\
			&
			\begin{matrix}
				\text{Lapl.}\\
				=\\
				l_1
			\end{matrix}
			\quad
			\frac{
				2
				\det
				\begin{bmatrix*}[r]
					r_3 & 1\\
					0   & -2
				\end{bmatrix*}
				-2
				\det
				\begin{bmatrix*}[r]
					3 & 1\\
					1 & -2
				\end{bmatrix*}
				+
				\det
				\begin{bmatrix*}[r]
					3 & r_3\\
					1 & 0
				\end{bmatrix*}
			}{
				7
			}\\
			&
			=
			\quad
			\frac{
				2
				(r_3 \cdot (-2))
				-2
				(3 \cdot (-2) - 1 \cdot 1)
				+
				(3 \cdot 0 - r_3 \cdot 1)
			}{
				7
			}\\
			&=
			\quad
			\frac{-5r_3 + 14}{7}
			\intertext{Dado que o número de aluno é \numeroEstudante, tem-se}
			r_3 &= \numeroEstudante \mod 1000 = 909\\
			x_4 &= \frac{-5 \cdot 909 + 14}{7} = -\frac{4531}{7} \approx
			-647.29
		\end{align*}
\end{enumerate}
