\grupo{}

\paragraph{} Tem-se $A \in \mathcal{M}_{5 \times 5}(\mathbb{K})$ e
$b \in \mathcal{M}_{5 \times 1}(\mathbb{K})$, definidas por:

\begin{align*}
	A &=
	\begin{bmatrix*}[r]
		2 & 1 & 0 & 3 & 0\\
		3 & -1 & 1 & 4 & 0\\
		0 & 0 & 2 & 1 & 1\\
		0 & 0 & 3 & 0 & 1\\
		0 & 0 & 1 & 0 & -2
	\end{bmatrix*},
	\quad
	b =
	\begin{bmatrix*}[r]
		0\\
		1\\
		2\\
		r_3\\
		0
	\end{bmatrix*}
\end{align*}


\begin{enumerate}[label=\alph*.]
	\item
		Pretende-se calcular o determinante de $A$, pelo que se vai recorrer
		múltiplas vezes ao Teorema de Laplace
		\parencite[pág. 129, Proposiçãp 3.9, Teorema de Laplace]{Cabral2012},
		no qual se utiliza a notação mencionada na bibliografia da cadeira.
		\begin{align*}
			\det A
			&=
			\det
			\begin{bmatrix*}[r]
				2 & 1 & 0 & 3 & 0\\
				3 & -1 & 1 & 4 & 0\\
				0 & 0 & 2 & 1 & 1\\
				0 & 0 & 3 & 0 & 1\\
				0 & 0 & 1 & 0 & -2
			\end{bmatrix*}\\
			&
			\begin{matrix}
				\text{Lapl.}\\
				=\\
				c_1
			\end{matrix}
			2
			\det
			\begin{bmatrix*}[r]
				-1 & 1 & 4 & 0\\
				0 & 2 & 1 & 1\\
				0 & 3 & 0 & 1\\
				0 & 1 & 0 & -2
			\end{bmatrix*}
			-
			3
			\det
			\begin{bmatrix*}[r]
				1 & 0 & 3 & 0\\
				0 & 2 & 1 & 1\\
				0 & 3 & 0 & 1\\
				0 & 1 & 0 & -2
			\end{bmatrix*}\\
			&
			\begin{matrix}
				\text{Lapl.}\\
				=\\
				c_1
			\end{matrix}
			-2
			\det
			\begin{bmatrix*}[r]
				2 & 1 & 1\\
				3 & 0 & 1\\
				1 & 0 & -2
			\end{bmatrix*}
			-
			3
			\det
			\begin{bmatrix*}[r]
				2 & 1 & 1\\
				3 & 0 & 1\\
				1 & 0 & -2
			\end{bmatrix*}\\
			&=
			- 5
			\det
			\begin{bmatrix*}[r]
				2 & 1 & 1\\
				3 & 0 & 1\\
				1 & 0 & -2
			\end{bmatrix*}\\
			&
			\begin{matrix}
				\text{Lapl.}\\
				=\\
				l_1
			\end{matrix}
			- 5\left(
			2
			\det
			\begin{bmatrix*}[r]
				0 & 1\\
				0 & -2
			\end{bmatrix*}
			-
			\det
			\begin{bmatrix*}[r]
				3 & 1\\
				1 & -2
			\end{bmatrix*}
			+
			\det
			\begin{bmatrix*}[r]
				3 & 0\\
				1 & 0
			\end{bmatrix*}\right)\\
			&=
			-5[-[3 \times (-2) - 1 \times 1]]
			= -35
		\end{align*}
		\clearpage
	\item\;\\
		Pretende-se calcular o valor de $x_4$.
		Como $\det A = -35 \neq 0$, podemos utilizar a Regra de
		Cramer
		\parencite[pãg. 152, Proposição 3.31(Regra de Cramer]{Cabral2012}.
		\begin{align*}
			x_4 = \frac{\det A_{(4)}}{\det A}
			&=
			-
			\frac{
				\det
				\begin{bmatrix*}[r]
					2 & 1  & 0 & 0   & 0\\
					3 & -1 & 1 & 1   & 0\\
					0 & 0  & 2 & 2   & 1\\
					0 & 0  & 3 & r_3 & 1\\
					0 & 0  & 1 & 5   & -2
				\end{bmatrix*}
			}{
				35
			}\\
			&
			\begin{matrix}
				\text{lapl.}\\
				=\\
				c_1
			\end{matrix}
			-
			\frac{
				2
				\det
				\begin{bmatrix*}[r]
					-1 & 1 & 1   & 0\\
					0  & 2 & 2   & 1\\
					0  & 3 & r_3 & 1\\
					0  & 1 & 5   & -2
				\end{bmatrix*}
				-
				3
				\det
				\begin{bmatrix*}[r]
					1  & 0 & 0   & 0\\
					0  & 2 & 2   & 1\\
					0  & 3 & r_3 & 1\\
					0  & 1 & 5   & -2
				\end{bmatrix*}
			}{
				35
			}\\
			&
			\begin{matrix}
				\text{Lapl.}\\
				=\\
				c_1,l_1
			\end{matrix}
			-
			\frac{
				-2
				\det
				\begin{bmatrix*}[r]
					2 & 2   & 1\\
					3 & r_3 & 1\\
					1 & 5   & -2
				\end{bmatrix*}
				-
				3
				\det
				\begin{bmatrix*}[r]
					2 & 2   & 1\\
					3 & r_3 & 1\\
					1 & 5   & -2
				\end{bmatrix*}
			}{
				35
			}\\
			&
			\begin{matrix}
				\text{Lapl.}\\
				=\\
				l_1
			\end{matrix}
			\frac{
				\det
				\begin{bmatrix*}[r]
					2 & 2   & 1\\
					3 & r_3 & 1\\
					1 & 5   & -2
				\end{bmatrix*}
			}{
				7
			}\\
			&
			\begin{matrix}
				\text{Lapl.}\\
				=\\
				l_1
			\end{matrix}
			\frac{
				2
				\det
				\begin{bmatrix*}[r]
					r_3 & 1\\
					5   & -2
				\end{bmatrix*}
				-
				2
				\det
				\begin{bmatrix*}[r]
					3 & 1\\
					1 & -2
				\end{bmatrix*}
				+
				\det
				\begin{bmatrix*}[r]
					3 & r_3\\
					1 & 5
				\end{bmatrix*}
			}{
				7
			}\\
			&=
			\frac{
				2
				(-2r_3 - 5)
				-
				2
				(-6 -1)
				+
				(15 - r_3)
			}{
				7
			}\\
			&=
			\frac{
				-5r_3 + 19
			}{
				7
			}
			\intertext{
				Dado que,
				$r_3 = \text{(número de aluno)}\mod 1000 = 2200909\mod 1000 = 909$,
				tem-se que
			}
			x_4 &= \frac{-5 \times 909 + 19}{7} = -646.57
		\end{align*}
\end{enumerate}
