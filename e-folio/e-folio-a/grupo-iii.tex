\grupo{}

\paragraph{\underline{Dados do enunciado}}

\paragraph{} Tem-se o seguinte sistema de equações:

\begin{equation}\label{eq-efa-g3}
	\begin{cases}
		ax + cy = 1\\
		-ax - 2y -az = -3\\
		2y + 2az = 2c
	\end{cases},
	\text{ com } a, c \in \mathbb{R}
\end{equation}

\paragraph{\underline{Resposta}}

\paragraph{}O sistema:

\begin{itemize}
	\item tem solução única para
		$(a, c) \in \mathbb{R}\backslash\{0\} \times
		\mathbb{R}\backslash\{1\}$;
	\item não tem solução para $(a, c) \in \mathbb{R}\backslash\{0\} \times \mathbb{R}
		\cup \mathbb{R} \times \mathbb{R}\backslash\{1\}$
\end{itemize}

\paragraph{\underline{Resolução}}

\paragraph{}O sistema de equações em \textbf{(\ref{eq-efa-g3})}, pode ser escrito na
forma $AX = b$, da seguinte forma:

\begin{align*}
	\begin{bmatrix*}[r]
		a  & c  & 0\\
		-a & -2 & -a\\
		0  & 2  & 2a
	\end{bmatrix*}&
	\begin{bmatrix*}[r]
		x\\
		y\\
		z
	\end{bmatrix*}
	=
	\begin{bmatrix*}[r]
		1\\
		-3\\
		2c
	\end{bmatrix*}
\end{align*}

\paragraph{} Calculemos agora o determinante da matriz simples:

\begin{align*}
	\det A
	&=
	\det
	\begin{bmatrix*}[r]
		a  & c  & 0\\
		-a & -2 & -a\\
		0  & 2  & 2a
	\end{bmatrix*}\\
	&
	\begin{matrix}
		\text{Lapl.}\\
		=\\
		c_1
	\end{matrix}
	a
	\det
	\begin{bmatrix*}[r]
		-2 & -a\\
		2  & 2a
	\end{bmatrix*}
	-
	(-a)
	\det
	\begin{bmatrix*}[r]
		c  & 0\\
		2  & 2a
	\end{bmatrix*}\\
	&
	=
	a\left(
		(-2) \cdot 2a
		- (-a) 2
	+
		2a \cdot c\right)\\
	&
	=
	2a^2\left(
		(-2)
		+ 1
	+
		c\right) = 2a^2(c - 1)
\end{align*}

\paragraph{}O sistema vai ter solução única tal que $\det A \neq 0$, para
$(a, c) \in \mathbb{R}\backslash\{0\} \times \mathbb{R}\backslash\{1\}$.

\begin{align*}
	\det A &= 0
	\iff 0 = 2a^2(c - 1)
	\iff a = 0 \quad \lor \quad  c = 1
\end{align*}

\paragraph{}Agora analisemos o caso $a = 0$. Voltando ao sistema definido em
\textbf{(\ref{eq-efa-g3})}, tem-se:

\begin{align*}
	\begin{cases}
		cy = 1\\
		- 2y = -3\\
		2y = 2c
	\end{cases}
	\iff
	\begin{cases}
		c = y^{-1}\\
		y = \frac{3}{2}\\
		y = c
	\end{cases}
	\iff
	\begin{cases}
		c = \frac{2}{3}\\
		y = \frac{3}{2}\\
		c = \frac{2}{3}
	\end{cases}
\end{align*}

\paragraph{}Pelo que se tem uma inconsistência, dado que $c$ não pode assumir
dois valores díspares ao mesmo tempo. Conclui-se que o sistema é impossível
para $a = 0$.

\paragraph{}Consideremos o caso $c = 1$:

\begin{align*}
	&
	\begin{cases}
		ax + y = 1\\
		-ax -2y -az = -3\\
		2y + 2az = 2
	\end{cases}
	\iff 
	\begin{cases}
		y = 1 - ax\\
		-ax -y -y -az = -3\\
		y = 1 - az
	\end{cases}\\
	\iff
	&
	\begin{cases}
		y = 1 - ax\\
		-ax -(1 - ax) -(1 - az) -az = -3\\
		y = 1 - az
	\end{cases}
	\iff
	\begin{cases}
		y = 1 - ax\\
		-2 = -3\\
		y = 1 - az
	\end{cases}
\end{align*}

\paragraph{}Dado que $-2 \neq -3$, porque o contrário seria uma
inconsistência, podemos concluir que o sistema é impossível, para $c = 1$.

