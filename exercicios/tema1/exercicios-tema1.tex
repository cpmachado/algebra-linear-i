\documentclass[11pt, a4paper]{article}
\usepackage[utf8]{inputenc}
\usepackage[portuguese]{babel}
\usepackage[T1]{fontenc}
\usepackage[margin=2cm]{geometry}
\usepackage{amsmath}
\usepackage{amsfonts}
\usepackage{amsthm}
\usepackage{mathtools}
\usepackage{csquotes}
\usepackage[citestyle=authoryear]{biblatex}
\usepackage[hidelinks]{hyperref}

\addbibresource{bibliografia.bib}

\title{
	Álgebra Linear I\\
	Exercícios do Tema 1: Matrizes\\
	Proposta de Resolução
}

\author{
	Carlos Pinto Machado
	<\href{mailto:2200909@estudante.uab.pt}{2200909@estudante.uab.pt}>
}

\hypersetup{
	pdfsubject = {Algebra Linear I},
	pdftitle = {Exercícios do Tema 1: Matrizes(Proposta de Resolução)},
	pdfauthor = {Carlos Pinto Machado(2200909)},
	pdfcreator = {},
	pdfproducer = {}
}

\newtheorem{proposicao}{Proposição}

\newcommand{\grupo}[1]{
	\begin{center}
		\bfseries \Large #1
	\end{center}
	\phantomsection
	\addcontentsline{toc}{section}{#1}
}

\newcommand{\exercicio}[1]{
	\begin{flushleft}
		\large #1.
	\end{flushleft}
	\phantomsection
	\addcontentsline{toc}{subsection}{#1.}
}

\newcommand{\alinea}[1]{
	\begin{flushleft}
		\normalsize (#1)
	\end{flushleft}
	\phantomsection
	\addcontentsline{toc}{subsubsection}{(#1)}
}


\begin{document}
\maketitle

\grupo{Operações com Matrizes}

\exercicio{1}

\begin{align*}
	&A =
	\begin{bmatrix}
		1 & 1 & 1\\
		1 & 1 & 1\\
		0 & 0 & 0
	\end{bmatrix},
	X =
	\begin{bmatrix*}
		1\\
		2\\
		3
	\end{bmatrix*},
	X^T =
	\begin{bmatrix*}
		1 & 2 & 3
	\end{bmatrix*},
	B =
	\begin{bmatrix*}
		6\\
		6\\
		0
	\end{bmatrix*},
	D =
	\begin{bmatrix}
		1 & 2 & 3\\
		0 & 4 & 5\\
		0 & 0 & 6
	\end{bmatrix},
	E =
	\begin{bmatrix}
		1 & 2 & 3\\
		0 & 4 & 5\\
		0 & 0 & 6\\
		0 & 0 & 0
	\end{bmatrix}, \\
	&F =
	\begin{bmatrix}
		1 & 0 & 0\\
		0 & 4 & 0\\
		0 & 0 & 6
	\end{bmatrix},
	G =
	\begin{bmatrix*}
		2 & 0 & 0\\
		0 & 2 & 0\\
		0 & 0 & 2
	\end{bmatrix*},
	H =
	\begin{bmatrix}
		1 & 0 & 0\\
		2 & 3 & 0\\
		4 & 0 & 6
	\end{bmatrix},
	I_3 =
	\begin{bmatrix}
		1 & 0 & 0\\
		0 & 1 & 0\\
		0 & 0 & 1
	\end{bmatrix},
	0_{3 \times 3} =
	\begin{bmatrix}
		0 & 0 & 0\\
		0 & 0 & 0\\
		0 & 0 & 0
	\end{bmatrix}
\end{align*}

\alinea{a}

\begin{itemize}
	\item matrizes quadradas: $A, D, F, G, H, I_3, 0_{3\times3}$
	\item matrizes triangulares superiores: $D, F, G, I_3, 0_{3\times3}$
	\item matrizes triangulares inferiores: $F, G, H, I_3, 0_{3\times3}$
	\item matrizes diagonais: $F, G, I_3, 0_{3\times3}$
	\item matrizes escalares: $G, I_3, 0_{3\times3}$
\end{itemize}

\clearpage

\alinea{b}

\paragraph{} As operações que não são possíveis são devido aos operandos serem
de classes de matriz diferentes.

\paragraph{}Estas são:

\begin{itemize}
	\item $A + X$, porque $A \in \mathcal{M}_{3 \times 3}(\mathbb{K})$ e
		$X \in \mathcal{M}_{3 \times 1}(\mathbb{K})$.
	\item $D + E$, porque $D \in \mathcal{M}_{3 \times 3}(\mathbb{K})$ e
		$E \in \mathcal{M}_{4 \times 3}(\mathbb{K})$.
\end{itemize}

\begin{align*}
	A + D
	&=
	\begin{bmatrix*}
		1 & 1 & 1\\
		1 & 1 & 1\\
		0 & 0 & 0
	\end{bmatrix*}
	+
	\begin{bmatrix*}
		1 & 2 & 3\\
		0 & 4 & 5\\
		0 & 0 & 6
	\end{bmatrix*}
	=
	\begin{bmatrix*}
		(1 + 1) & (1 + 2) & (1 + 3)\\
		(1 + 0) & (1 + 4) & (1 + 5)\\
		(0 + 0) & (0 + 0) & (0 + 6)
	\end{bmatrix*}
	=
	\begin{bmatrix*}
		2 & 3 & 4\\
		1 & 5 & 6\\
		0 & 0 & 6
	\end{bmatrix*}\\
	2A + D
	&= A + (A + D)
	=
	\begin{bmatrix*}
		1 & 1 & 1\\
		1 & 1 & 1\\
		0 & 0 & 0
	\end{bmatrix*}
	+
	\begin{bmatrix*}
		2 & 3 & 4\\
		1 & 5 & 6\\
		0 & 0 & 6
	\end{bmatrix*}
	=
	\begin{bmatrix*}
		(1 + 2) & (1 + 3) & (1 + 4)\\
		(1 + 1) & (1 + 5) & (1 + 6)\\
		(0 + 0) & (0 + 0) & (0 + 6)
	\end{bmatrix*}
	=
	\begin{bmatrix*}
		3 & 4 & 5\\
		2 & 6 & 7\\
		0 & 0 & 6
	\end{bmatrix*} \\
	2(A + D)
	&=
	2
	\begin{bmatrix*}
		2 & 3 & 4\\
		1 & 5 & 6\\
		0 & 0 & 6
	\end{bmatrix*}
	=
	\begin{bmatrix*}[r]
		4 & 6  & 8\\
		2 & 10 & 12\\
		0 & 0  & 12
	\end{bmatrix*}\\
	2A + 2D
	&= 2(A + D)
	=
	\begin{bmatrix*}[r]
		4 & 6  & 8\\
		2 & 10 & 12\\
		0 & 0  & 12
	\end{bmatrix*}\\
	D + F
	&=
	\begin{bmatrix*}
		1 & 2 & 3\\
		0 & 4 & 5\\
		0 & 0 & 6
	\end{bmatrix*}
	+
	\begin{bmatrix*}
		1 & 0 & 0\\
		0 & 4 & 0\\
		0 & 0 & 6
	\end{bmatrix*}
	+
	\begin{bmatrix*}
		(1 + 1) & (2 + 0) & (3 + 0)\\
		(0 + 0) & (4 + 4) & (5 + 0)\\
		(0 + 0) & (0 + 0) & (6 + 6)
	\end{bmatrix*}
	=
	\begin{bmatrix*}[r]
		2 & 2 & 3\\
		0 & 8 & 5\\
		0 & 0 & 12
	\end{bmatrix*}
\end{align*}

\alinea{c}

\begin{align*}
	G
	=
	\begin{bmatrix*}
		2 & 0 & 0\\
		0 & 2 & 0\\
		0 & 0 & 2
	\end{bmatrix*}
	&=
	2
	\begin{bmatrix*}
		1 & 0 & 0\\
		0 & 1 & 0\\
		0 & 0 & 1
	\end{bmatrix*}
	=
	2 \; I_3\\
	\begin{bmatrix*}
		1\\
		1\\
		0
	\end{bmatrix*}
	+ 2
	\begin{bmatrix*}
		1\\
		1\\
		0
	\end{bmatrix*}
	+ 3
	\begin{bmatrix*}
		1\\
		1\\
		0
	\end{bmatrix*}
	&=
	(1 + 2 + 3)
	\begin{bmatrix*}
		1\\
		1\\
		0
	\end{bmatrix*}
	=
	\begin{bmatrix*}
		6\\
		6\\
		0
	\end{bmatrix*}
	= B
\end{align*}

\alinea{d}

\begin{align*}
	&&2A + 3Y - G &= 0_{3 \times 3} \\
	&\iff& Y &= \frac{1}{3}\left(G - 2A\right)\\
	&&&=
	\frac{1}{3}
	\left(
	\begin{bmatrix*}
		2 & 0 & 0\\
		0 & 2 & 0\\
		0 & 0 & 2
	\end{bmatrix*}
	-
	\begin{bmatrix*}
		2 & 2 & 2\\
		2 & 2 & 2\\
		0 & 0 & 0\\
	\end{bmatrix*}\right)\\
	&&&=
	\frac{1}{3}
	\begin{bmatrix*}[r]
		0            & -2 & -2\\
		-2           & 0  & -2\\
		0            & 0  & 2
	\end{bmatrix*}\\
	&&&=
	\begin{bmatrix*}[r]
		0            & -\frac{2}{3} & -\frac{2}{3}\\
		-\frac{2}{3} & 0            & -\frac{2}{3}\\
		0            & 0            & \frac{2}{3}
	\end{bmatrix*}
\end{align*}

\clearpage

\exercicio{2}

\begin{align*}
	(DH)_{3,2}
	&=
	\begin{bmatrix*}
		0 & 0 & 6
	\end{bmatrix*}
	\begin{bmatrix*}
		0\\
		3\\
		0
	\end{bmatrix*}
	= 0 \cdot 0 + 0 \cdot 3 + 6 \cdot 0 = 0\\
	DH
	&=
	\begin{bmatrix}
		1 & 2 & 3\\
		0 & 4 & 5\\
		0 & 0 & 6
	\end{bmatrix}
	\begin{bmatrix}
		1 & 0 & 0\\
		2 & 3 & 0\\
		4 & 0 & 6
	\end{bmatrix}
	=
	\begin{bmatrix*}[r]
		17 & 6  & 18\\
		28 & 12 & 30\\
		24 & 0  & 36
	\end{bmatrix*}
	\\
	HD
	&=
	\begin{bmatrix}
		1 & 0 & 0\\
		2 & 3 & 0\\
		4 & 0 & 6
	\end{bmatrix}
	\begin{bmatrix}
		1 & 2 & 3\\
		0 & 4 & 5\\
		0 & 0 & 6
	\end{bmatrix}
	=
	\begin{bmatrix*}[r]
		1 & 2  & 3\\
		2 & 16 & 21\\
		4 & 8  & 48
	\end{bmatrix*}
\end{align*}


\exercicio{3}

\paragraph{} Produtos de matrizes que não são possíveis devido às colunas do
primeiro operando divergir do número de linhas do segundo operando:

\begin{itemize}
	\item $X A$
	\item $A X^T$
	\item $D E$
\end{itemize}

\begin{align*}
	A X
	&=
	\begin{bmatrix*}
		1 & 1 & 1\\
		1 & 1 & 1\\
		0 & 0 & 0
	\end{bmatrix*}
	\begin{bmatrix*}
		1\\
		2\\
		3
	\end{bmatrix*}
	=
	\begin{bmatrix*}
		6\\
		6\\
		0
	\end{bmatrix*}\\
	X^T A
	&=
	\begin{bmatrix*}
		1 & 2 & 3
	\end{bmatrix*}
	\begin{bmatrix*}
		1 & 1 & 1\\
		1 & 1 & 1\\
		0 & 0 & 0
	\end{bmatrix*}
	=
	\begin{bmatrix*}
		3 & 3 & 3
	\end{bmatrix*}
	\\
	E D
	&=
	\begin{bmatrix*}
		1 & 2 & 3\\
		0 & 4 & 5\\
		0 & 0 & 6\\
		0 & 0 & 0
	\end{bmatrix*}
	\begin{bmatrix*}
		1 & 2 & 3\\
		0 & 4 & 5\\
		0 & 0 & 6
	\end{bmatrix*}
	=
	\begin{bmatrix*}[r]
		1 & 10 & 31\\
		0 & 16 & 50\\
		0 & 0  & 36\\
		0 & 0  & 0
	\end{bmatrix*}
	\\
	X X^T
	&=
	\begin{bmatrix*}
		1\\
		2\\
		3
	\end{bmatrix*}
	\begin{bmatrix*}
		1 & 2 & 3
	\end{bmatrix*}
	=
	\begin{bmatrix*}
		1 & 2 & 3\\
		2 & 4 & 6\\
		3 & 6 & 9\\
	\end{bmatrix*}
	\\
	X^T X
	&=
	\begin{bmatrix*}
		1 & 2 & 3
	\end{bmatrix*}
	\begin{bmatrix*}
		1\\
		2\\
		3
	\end{bmatrix*}
	=
	\begin{bmatrix*}
		14
	\end{bmatrix*}
	\\
	AG &= A \cdot 2 I_3 = 2 A I_3 = 2 A =
	\begin{bmatrix*}
		2 & 2 & 2\\
		2 & 2 & 2\\
		0 & 0 & 0
	\end{bmatrix*}
	\\
	GA &= 2 I_3 A = 2 A=
	\begin{bmatrix*}
		2 & 2 & 2\\
		2 & 2 & 2\\
		0 & 0 & 0
	\end{bmatrix*}
\end{align*}

\begin{align*}
	AF
	&=
	\begin{bmatrix*}
		1 & 1 & 1\\
		1 & 1 & 1\\
		0 & 0 & 0
	\end{bmatrix*}
	\begin{bmatrix*}
		1 & 0 & 0\\
		0 & 4 & 0\\
		0 & 0 & 6
	\end{bmatrix*}
	=
	\begin{bmatrix*}
		1 & 4 & 6\\
		1 & 4 & 6\\
		0 & 0 & 0
	\end{bmatrix*}
	\\
	FA
	&=
	\begin{bmatrix*}
		1 & 0 & 0\\
		0 & 4 & 0\\
		0 & 0 & 6
	\end{bmatrix*}
	\begin{bmatrix*}
		1 & 1 & 1\\
		1 & 1 & 1\\
		0 & 0 & 0
	\end{bmatrix*}
	=
	\begin{bmatrix*}
		1 & 1 & 1\\
		4 & 4 & 4\\
		0 & 0 & 0
	\end{bmatrix*}
	\\
	A I_3 &= A\\
	I_3 A &= A\\
	D^2
	&=
	\begin{bmatrix*}
		1 & 2 & 3\\
		0 & 4 & 5\\
		0 & 0 & 6
	\end{bmatrix*}
	\begin{bmatrix*}
		1 & 2 & 3\\
		0 & 4 & 5\\
		0 & 0 & 6
	\end{bmatrix*}
	=
	\begin{bmatrix*}[r]
		1 & 10  & 31\\
		0 & 16 & 50\\
		0 & 0  & 36
	\end{bmatrix*}
	\\
	H^2
	&=
	\begin{bmatrix*}
		1 & 0 & 0\\
		2 & 3 & 0\\
		4 & 0 & 6
	\end{bmatrix*}
	\begin{bmatrix*}
		1 & 0 & 0\\
		2 & 3 & 0\\
		4 & 0 & 6
	\end{bmatrix*}
	=
	\begin{bmatrix*}[r]
		1  & 0 & 0\\
		8  & 9 & 0\\
		28 & 0 & 36
	\end{bmatrix*}
	\\
	F^2
	&=
	\begin{bmatrix*}
		1 & 0 & 0\\
		0 & 4 & 0\\
		0 & 0 & 6
	\end{bmatrix*}
	\begin{bmatrix*}
		1 & 0 & 0\\
		0 & 4 & 0\\
		0 & 0 & 6
	\end{bmatrix*}
	=
	\begin{bmatrix*}[r]
		1 & 0 & 0\\
		0 & 16 & 0\\
		0 & 0 & 36
	\end{bmatrix*}
	\\
	F^3
	&= F^2 F =
	\begin{bmatrix*}[r]
		1 & 0 & 0\\
		0 & 16 & 0\\
		0 & 0 & 36
	\end{bmatrix*}
	\begin{bmatrix*}
		1 & 0 & 0\\
		0 & 4 & 0\\
		0 & 0 & 6
	\end{bmatrix*}
	=
	\begin{bmatrix*}[r]
		1 & 0 & 0\\
		0 & 64 & 0\\
		0 & 0 & 216
	\end{bmatrix*}
\end{align*}

\exercicio{4}

\begin{align*}
	A =
	\begin{bmatrix*}
		1\\
		2
	\end{bmatrix*} \in \mathcal{M}_{2 \times 1}(\mathbb{K}),
	B =
	\begin{bmatrix*}
		1 & 0\\
		2 & 1\\
		0 & 1
	\end{bmatrix*}\in \mathcal{M}_{3 \times 2}(\mathbb{K}),
	C =
	\begin{bmatrix*}[r]
		3 & 2 & 0\\
		1 & 2 & -1
	\end{bmatrix*}\in \mathcal{M}_{2 \times 3}(\mathbb{K})
\end{align*}

\paragraph{} Dado que $A$ tem um número de colunas diferentes do número de
linhas de $B$ e $C$, os produtos $AB$ e $AC$ não são possíveis. Deste modo,
teremos um produto $BC$ ou $CB$ que será multiplicado por $A$ à direita. Pelo
mesmo raciocínio podemos excluir o produto $CA$, pelo que forçosamente
será o produto $CBA$.

\begin{align*}
	CBA =
	\begin{bmatrix*}[r]
		3 & 2 & 0\\
		1 & 2 & -1
	\end{bmatrix*}
	\begin{bmatrix*}
		1 & 0\\
		2 & 1\\
		0 & 1
	\end{bmatrix*}
	\begin{bmatrix*}
		1\\
		2
	\end{bmatrix*}
	=
	\begin{bmatrix*}
		7 & 2\\
		5 & 1
	\end{bmatrix*}
	\begin{bmatrix*}
		1\\
		2
	\end{bmatrix*}
	=
	\begin{bmatrix*}
		11\\
		7
	\end{bmatrix*}
\end{align*}

\clearpage

\exercicio{5}

\begin{proposicao}\label{prop:tema-1-ex5-1}
	\begin{align*}
		\forall a, b, c \in \mathbb{K},
		\begin{bmatrix*}
			a & b & c
		\end{bmatrix*}
		\begin{bmatrix*}
			1.1 & 1.2 & 1.3\\
			2.1 & 2.2 & 2.3\\
			3.1 & 3.2 & 3.3
		\end{bmatrix*}
		=
		a
		\begin{bmatrix*}
			1.1 & 1.2 & 1.3
		\end{bmatrix*}
		+
		b
		\begin{bmatrix*}
			2.1 & 2.2 & 2.3
		\end{bmatrix*}
		+
		c
		\begin{bmatrix*}
			3.1 & 3.2 & 3.3
		\end{bmatrix*}
	\end{align*}
\end{proposicao}

\begin{proof}
	\begin{align*}
		\begin{bmatrix*}
			a & b & c
		\end{bmatrix*}
		\begin{bmatrix*}
			1.1 & 1.2 & 1.3\\
			2.1 & 2.2 & 2.3\\
			3.1 & 3.2 & 3.3
		\end{bmatrix*}
	\end{align*}
\end{proof}


\begin{proposicao}\label{prop:tema-1-ex5-2}
	\begin{align*}
		\forall a, b, c \in \mathbb{K},
		\begin{bmatrix*}
			1.1 & 1.2 & 1.3\\
			2.1 & 2.2 & 2.3\\
			3.1 & 3.2 & 3.3
		\end{bmatrix*}
		\begin{bmatrix*}
			a\\
			b\\
			c
		\end{bmatrix*}
		=
		a
		\begin{bmatrix*}
			1.1\\
			2.1\\
			3.1
		\end{bmatrix*}
		+
		b
		\begin{bmatrix*}
			1.2\\
			2.2\\
			3.2
		\end{bmatrix*}
		+
		c
		\begin{bmatrix*}
			1.3\\
			2.3\\
			3.3
		\end{bmatrix*}
	\end{align*}
\end{proposicao}

\begin{proof}
	\begin{align*}
		\begin{bmatrix*}
			1.1 & 1.2 & 1.3\\
			2.1 & 2.2 & 2.3\\
			3.1 & 3.2 & 3.3
		\end{bmatrix*}
		\begin{bmatrix*}
			a\\
			b\\
			c
		\end{bmatrix*}
	\end{align*}
\end{proof}



\clearpage

\exercicio{6}

\begin{align*}
	\begin{bmatrix*}
		3 & 0 & 0\\
		0 & 1 & 0\\
		0 & 0 & 1
	\end{bmatrix*}
	\begin{bmatrix*}
		1.1 & 1.2 & 1.3\\
		2.1 & 2.2 & 2.3\\
		3.1 & 3.2 & 3.3
	\end{bmatrix*}
	=
	\begin{bmatrix*}
		3.3 & 3.6 & 3.9\\
		2.1 & 2.2 & 2.3\\
		3.1 & 3.2 & 3.3
	\end{bmatrix*}
\end{align*}


\exercicio{7}

\begin{align*}
	\begin{bmatrix*}
		1 & 0 & 0\\
		0 & 1 & 0\\
		3 & 0 & 1
	\end{bmatrix*}
	\begin{bmatrix*}
		1.1 & 1.2 & 1.3\\
		2.1 & 2.2 & 2.3\\
		3.1 & 3.2 & 3.3
	\end{bmatrix*}
	=
	\begin{bmatrix*}
		1.1 & 1.2 & 1.3\\
		2.1 & 2.2 & 2.3\\
		6.4 & 6.8 & 7.2
	\end{bmatrix*}
\end{align*}

\exercicio{8}

\exercicio{9}

\alinea{a}

\alinea{b}

\alinea{c}

\alinea{d}

\exercicio{10}

\exercicio{11}

\exercicio{12}

\exercicio{13}


\clearpage

\nocite{*}
\printbibliography[title={Bibliografia},heading=bibintoc]

\end{document}
